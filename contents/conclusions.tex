% !TEX root = ../main.tex

\chapter{结论}

\section{本文工作总结}

本论文围绕“面向用户交互的在线实时数字人驱动”这一核心目标,针对现有语音驱动手势生成方法普遍依赖整句输入、难以满足严格实时约束的问题,提出了一种仅使用在线可获取信号(语音、面部表情与头部姿态)即可逐帧生成上半身骨骼动作的帧级生成方法 FaceCapGes,实现了无需用户实际做出手势的低门槛数字人自然表达能力。论文的主要工作与贡献可总结如下:

(1)构建了面向在线实时数字人驱动的多模态帧级手势生成系统框架。
针对在线交互场景下语音逐字输入带来的未来信息不可用问题,本文从系统层面提出了严格因果的实时生成机制,明确了从信号采集、特征同步、帧级推理到虚拟人驱动渲染的完整流水线结构,并在实时约束下制定了输入模态选择、姿态表示方式以及端到端数据流组织策略。该框架保证了模型能够在仅依赖当前与历史信息的情况下持续输出动作流,为后续的可部署实现提供了系统基础与设计规范。

(2)在级联多模态架构中引入头部姿态作为新的实时输入模态,并提出弱耦合融合策略以增强节奏与指向一致性。
本文在继承 CaMN 级联设计思想的基础上,将头部姿态视为实时可获取的辅助输入信号,用于提供节奏前瞻与空间锚定线索,从而弥补语音模态在严格因果条件下对方向一致性与互动焦点建模不足的问题。为此,论文设计了头部姿态编码器,并通过弱耦合方式将其作为独立通道拼接进多模态隐向量,避免其与语音/表情编码产生过强耦合导致收敛困难。该设计在保持在线实时推理能力的同时,使模型能够显式利用用户头部朝向变化,从而增强生成动作的空间协调性与指向一致性。

(3)提出并实现了基于滑动窗口展开的闭环自回归训练策略,通过在片段内部逐帧展开单步因果预测器并写回历史动作条件,使模型在自身生成历史上学习稳定的连续输出,从而提升实时生成的时间一致性并缓解自回归漂移与抖动问题。

(4)构建了完整的评估平台并进行了主客观与性能实验验证,证明 FaceCapGes 在严格因果条件下仍具备良好的生成质量与实时推理能力。
论文搭建了统一的跨模型评估与渲染平台,使不同方法可在相同输入条件与渲染设置下公平对比,并在 BEAT 数据集上对 FaceCapGes、CaMN 与 DiffSHEG 等代表性方法开展了用户主观评估、客观指标测量,并进一步对FaceCapGes进行了消融分析与实时性能测试。实验结果表明,在严格因果约束下,FaceCapGes 的生成真实性可达到与扩散模型方法相当的水平;同时在韵律变化较强的语音片段及头部方向变化明显的场景中,其动作表现更平滑且空间指向与真实数据更一致,体现出该方法在虚拟交互场景中保持空间表达一致性的优势。此外,模型帧级推理效率与端到端链路延迟满足实时交互应用的运行需求,验证了其作为可部署在线数字人驱动方案的可行性。

综上所述,本论文在严格因果约束下,提出了基于语音、面部捕捉与头部姿态的在线帧级手势生成方法 FaceCapGes,并从系统设计、模型结构、训练策略与实验验证四个方面证明了在无需未来信息的条件下实现自然随语手势生成的可行性与应用价值。FaceCapGes 在严格实时约束下实现了无需手势采集的自然随语动作生成,可支持实时虚拟人直播、沉浸式虚拟社交互动等交互式数字人应用场景,提升了数字人表达的自然性与互动一致性。与此同时,本研究也为未来进一步融合高层语义信息与预测性控制机制的实时数字人驱动方法提供了技术基础与参考。

\section{未来工作展望}

\subsection{高层语义信息}

当前模型主要关注语音声学特征与运动感知模态对手势生成的影响,尚未显式引入语言层面的语义理解与表达意图建模。未来可结合实时语音识别与增量式语义解析技术,引入语篇结构、强调意图或对话功能等高层语义信息,以丰富手势在交互场景中的表达能力。在不破坏实时性的前提下,探索对有延迟但可修正的语义假设的鲁棒利用方式,将有助于提升生成手势在语义层面的准确性与一致性。

\subsection{面向未来趋势的预测性训练目标}

从建模目标的角度来看,当前 FaceCapGes 的训练过程主要以当前时间步手势姿态的重建误差为优化目标,即在给定历史与当前多模态输入的条件下,监督模型对当前手势的预测精度。然而,该学习目标并未对未来时间段内手势节奏与结构变化施加显式约束,使模型对历史信息的利用更多服务于当前帧生成,而非对即将发生的动作变化进行前瞻性建模。

未来的研究可在现有框架基础上,引入针对手势未来趋势的预测性监督信号,尤其是充分挖掘头部与面部动态中所蕴含的准备性线索。与直接预测未来具体手势姿态不同,该方向更侧重于对抽象化时序属性的建模,例如未来短时间窗口内的手势起始概率、运动能量变化或强调强度等。这类趋势性变量具有时间平滑、语义明确且可提前出现的特点,适合作为实时系统中的前瞻性约束。

通过在训练阶段同时优化当前手势生成与未来趋势预测两个目标,模型有望学习到更具时间结构性的中间表示,从而在不引入额外模态或显著增加系统延迟的前提下,实现对手势节奏的提前准备与更稳定的时序对齐。