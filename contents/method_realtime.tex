\chapter{基于滑动窗口的实时手势生成自回归训练}

\section{因果时序建模与训练策略}
\label{sec:realtime}
\subsection{时间建模结构改动与因果性约束}

基线模型CaMN使用双向LSTM生成完整序列的骨骼姿态,输入与输出片段长度一致。
由于双向结构在每个时间步都依赖未来帧隐状态,虽然能增强整体平滑性,但不满足实时生成场景的因果约束。
为实现严格的实时性,本文将时间建模模块改为单向LSTM,
使模型在每一时间步仅依赖过去$N$帧的输入并预测当前帧的骨骼姿态。
虽然单向LSTM结构上仍会输出与输入片段等长的时间序列,
但有效输出为输出序列的最后一帧(即当前时刻$t$),
我们对此计算重构与平滑损失。
形式化地,令
\begin{equation}
\hat{\bm{v}}_{t} = \mathrm{LSTM}(\bm{z}_{t-N:t}^{fuse})_{N},
\end{equation}
其中,下标$N$表示取LSTM输出序列的最后一帧作为当前时刻的预测结果。

则前述损失函数中的姿态、速度与加速度项均以该目标帧为中心计算,
即
\begin{equation}
\mathcal{L}_{rec} = \mathcal{L}_{Huber}(\bm{v}_t,\hat{\bm{v}}_t),
\end{equation}

其余项 $\mathcal{L}_{vel},\mathcal{L}_{acc}$ 亦同理,由 $\hat{\bm{v}}_t$ 与历史帧差分得到。
最终仍采用式\ref{eq:loss_total}定义的总体目标进行优化。

该策略通过在前$N$帧内累积隐状态,完成当前姿态预测,从而建立严格的因果时序映射。

在推理阶段,FaceCapGes采用长度为$N$的显式输入窗口,并在时间步之间保留LSTM的隐状态。
虽然单向LSTM理论上能够仅通过递推隐状态存储历史信息,
但由于隐状态为压缩形式,难以完全保留短时节奏与相位特征。
因此,显式窗口输入与隐状态记忆在模型中形成互补:
前者提供局部的高分辨率上下文,
后者维持全局的时序连贯性。
这种设计在保证因果性的前提下提高了生成的稳定性与自然性,
也是实现实时语音驱动动作生成的关键因素之一。

图~\ref{fig_lstmcompare}展示了双向与单向结构的差异。
双向结构在每个时间步同时利用历史与未来帧特征进行建模;
而单向结构仅基于历史帧进行递推,以保持因果性并支持流式推理,
因此仅依赖历史帧输入,更适合在流式序列中逐帧输出预测结果。

\begin{figure}[h!t]
\centering
\includegraphics[width=0.9\linewidth]{figures/Fig_lstmcompare.png}
\bicaption{双向与单向 LSTM 的因果性对比示意图}{Illustration of Bidirectional vs. Unidirectional LSTM under Causality Constraint}
\label{fig_lstmcompare}
\end{figure}

\subsection{滑动窗口式自回归训练}

训练时,每段输入为$N+M$帧,前$N$帧作为输入上下文,后$M$帧逐步预测(见图~\ref{fig_slideWindow})。
其中,前$N$帧的历史姿态可视为模型的因果历史窗口,
其设计思路与基线模型中附加片段首部的“历史缓冲”类似,
但在意义上有所不同。
CaMN在双向时间建模下使用该缓冲以补足片段外部上下文,
而FaceCapGes则将其重新定义为实时预测所需的前序姿态帧数量,
即当前帧推理所依赖的显式时间上下文。

在训练阶段,模型采用纯自回归(pure autoregressive)方式展开,
即在每一步预测后,将自身生成的历史帧作为下一步输入,
而非采用教师强制(teacher forcing)。
这种方式使模型在优化过程中暴露于自身预测的分布,
保持训练与推理过程的一致性,
避免了教师强制常见的暴露偏差(exposure bias),
即推理阶段模型面对自身生成数据时性能下降的问题。
在每个窗口内连续预测 $M$ 步后,
对所有预测帧计算式\ref{eq:loss_total}定义的损失,
并在时间维度上取平均作为该窗口的优化目标:
\begin{equation}
\mathcal{L}_{window}
=
\frac{1}{M}\sum_{i=1}^{M}
\mathcal{L}_{total}^{(i)},
\end{equation}
其中 $\mathcal{L}_{total}^{(i)}$ 表示在第 $i$ 帧处依据式\ref{eq:loss_total}计算得到的总损失。

纯自回归训练使模型在遇到自身预测误差时能够动态修正节奏,
从而在长时间交互中保持自然的动作连续性与节奏稳定性。

每一预测步中定义一个长度为$N+1$的滑动窗口,
其中前$N$帧为输入,第$N+1$帧为预测目标。
形式化定义如下:
\begin{align}
\bm{g}^H_i &= (\bm{g}_{i-N}, ..., \bm{g}_{\min(N, i-1)}) \otimes (\hat{\bm{g}}_{\max(N+1, i-N)}, ..., \hat{\bm{g}}_{i-1}), \\
\hat{\bm{g}}_i &= FaceCapGes(\bm{v}_{i-N}, ..., \bm{v}_i; \bm{g}^H_i),
\end{align}
其中$\otimes$表示时间拼接操作。
该机制在每步仅依赖过去信息,从而保持因果性约束;
同时通过窗口内的滚动更新,在不引入未来帧的前提下实现平滑过渡。
推理阶段模型以单帧为输入流,输出当前时刻的上半身姿态,实现端到端低延迟生成。

需要指出的是,由于滑动窗口机制依赖前$N$帧的上下文信息,
模型在序列开端无法立即生成动作,即存在一个短暂的冷启动阶段。
然而,在本文的目标应用场景中,
模型作为常驻进程伴随用户的虚拟人交互持续运行,无需在每一句发话时切换启动状态。
因此该延迟仅在首次启动程序时出现$N$帧的等待,
对用户体验影响可基本忽略。

\begin{figure}[h!t]
\centering
\includegraphics[width=0.75\linewidth]{figures/Fig2.png}
\bicaption{滑动窗口训练示意图}{Sliding-window Training Illustration}
\label{fig_slideWindow}
\end{figure}

通过以上适配,FaceCapGes在保持CaMN级联优势的同时显著降低系统延迟,
实现实时稳定的语音与面部驱动手势生成。

\section{本章小结}