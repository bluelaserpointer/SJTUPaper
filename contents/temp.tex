
\section{系统设计思路}

\subsection{手势生成目标的选择}
手势生成任务可根据其对语义上下文的依赖程度与时间结构的复杂度,
大致划分为以语义理解为核心的“语义驱动型生成”,
与以韵律对齐为核心的“韵律驱动型生成”。
本文面向用户实时数字人驱动场景,在严格因果与低延迟约束下,
模型无法访问未来语音或完整文本语义,因此难以稳定生成对语义推理要求较高的形象性、隐喻性与指示性手势。
基于该约束,本文将研究重点聚焦于与语音韵律高度同步的节奏型手势生成,并将其视为在实时交互条件下具有明确表达价值且可稳定建模的随语手势形式。

在此定位下,本文提出帧级多模态级联手势生成模型 FaceCapGes,输入为实时语音特征、面部 BlendShape 权重以及头部姿态旋转参数,并在不依赖未来帧信息的条件下逐帧输出上半身骨骼姿态。模型通过多模态信号融合弥补单一语音模态在实时场景下的信息不足:面部表情提供情绪与说话强度等辅助线索,头部姿态提供节奏前瞻与空间锚定信号,从而在保持可部署实时性的同时提升生成动作的自然度、同步性与方向一致性。

综合实时数字人驱动场景的交互需求与部署约束,本文系统设计遵循以下原则:

(1) 严格因果性: 模型仅使用当前与过去的多模态输入,不访问未来帧信息;

(2) 低延迟: 支持帧级在线推理,满足实时交互的时延要求;

(3) 可实时采集模态: 输入模态需能通过常见设备实时获取,包括语音、面部参数与头部姿态;

(4) 可稳定建模: 在上述约束下优先建模节奏型手势,并将其作为核心生成目标;

\subsection{节奏型手势在随语手势中的重要性}
尽管节奏型手势通常不承载具体语义信息,%
已有研究表明,其在交流效果与听众感知层面仍具有独立的价值。%
Baars等的实验\cite{FlapThoseHands}比较了无手势、仅使用节奏型手势、以及包含了形象性、隐喻性、指示性的意义性手势的三种演讲条件,%
结果显示,相较于完全不做手势,仅使用节奏型手势即可显著提升听众对说话者自然度的主观评价,并一定程度上提升了听众对演讲内容的记忆表现。%
而包含意义性手势的演讲条件在自然度与听众的记忆保留等指标上并未显著优于节奏型手势条件。
这一发现表明,即便缺乏形象性或隐喻性的语义映射,节奏型手势仍能通过与语音韵律的同步,对交流过程产生积极影响。

从功能上看,节奏型手势主要服务于语篇结构与韵律组织,%
其作用并非传递附加语义,而是通过时间对齐、重音标记与注意力引导,%
增强语音信息的感知显著性与节奏感。%
在真实的人机交互与虚拟人系统中,
这类手势常被作为一种低语义依赖、但高度稳健的非语言表达形式加以采用。

鉴于本文面向低延迟、严格逐帧的在线驱动场景,%
系统在任一时间步均无法获取未来文本或完整语义结构,%
对语义一致性要求较高的形象性与隐喻性手势难以可靠生成。%
相比之下,节奏型手势主要依赖于当前及局部时间窗口内的语音韵律特征,%
更适合在实时条件下进行稳定建模与生成。%
因此,本文选择以节奏型手势作为主要研究对象,%
并将其视为一种在系统约束下具有明确交互价值的可行随语手势形式。

\subsection{引入头部姿态的动机与贡献}
从生成可行性的角度,现有研究普遍认为节奏型手势可在无语义理解的条件下由语音韵律直接驱动生成。
多数语音驱动手势研究证实,仅凭语音的能量、时长与音高变化即可合成自然的节奏性上肢动作\cite{ginosar2019speech2gesture,alexanderson2020stylegestures,kucherenko2021movingfastslow}。
这些研究所生成的动作在时间结构上与语音重音同步,体现了语音与手势共享的时间规划机制。

相比之下,iconic(形象性)、metaphoric(隐喻性)与deictic(指向性)手势均依赖语义或指向关系,
需要从上下文分析语义与语境,难以在严格实时的因果条件下生成。
而节奏相关特征在音频中则具有更高的可预测性。\cite{kucherenko2021predictability}
这表明,在缺乏未来语义与全局上下文的实时场景中,仅凭语音模态,模型只能稳定生成节奏层面的动作。

为突破这一限制,本文引入头部姿态模态作为补充输入信号。头部姿态能在实时因果条件下提供部分空间与时间线索:其转头与注视方向反映互动焦点,点头与抬头与语音重读共现,能够在不依赖未来语义信息的前提下,为手势生成提供弱先验约束。
这种模态扩展为实时系统提供了理论上的可行性基础,使模型能够在语音之外获得关于节奏、方向与视角的附加信息。

\subsubsection{头部姿态对手势预测的贡献}
头部动作在自然语音中常呈现出一定的时间前瞻性~\cite{esteve2017timing}:%
其启动往往早于对应韵律词的发声,%
这意味着视觉模态可能比声学信号更早反映语音节奏的变化趋势。%
这种时序特性为实时生成任务提供了潜在的预测窗口,%
使系统能够在语音节奏变化尚未显现前,就提前捕获相关的动态线索。%
因此,头部姿态在实时生成中不仅提供同步参考,也可能在时间上形成前驱信号,为手势节奏的自然启动提供时序优势。%

\subsubsection{头部姿态对空间锚定与视角一致性的贡献}
头部姿态模态为实时语音驱动的手势生成提供了关键的空间参照信号。%
其与语音韵律在时间组织上高度耦合。%
即使在无未来语义信息的条件下,头部的转向与注视变化仍能反映说话者的注意焦点与叙述方向,%
从而帮助模型在动作生成中保持空间的连贯性与方向一致性。%
这一机制使系统能够在时间与空间两个维度上同步对齐语音与动作,%
让生成的手势在视觉上更具互动感与表达意图。

在McNeill的四类手势体系中,头部姿态的引入主要强化了两类动作的生成:  

(1) 对beat手势而言,它为语音重读和节奏段落提供显式的时间协同信号,使手部与头部动作在韵律层面更加一致;  

(2) 对iconic手势而言,它在具有路径与方向特征的动作中提供空间参考,使模型能够在叙事空间中更稳定地确定动作的方位与轨迹方向。  

通过这两方面的强化,系统在保持实时性的同时获得了更自然的节奏衔接与空间表达。

与此同时,本文明确头部姿态模态的作用边界:其核心优势在于捕捉方向、焦点与时序节奏,而非手型语义或复杂形态描摹等细粒度语义特征。换言之,它主要改善手势的位置、方向与视角依附,而非手势的形状描绘或语义内容。对于依赖抽象语义或外指参照的metaphoric与deictic手势,仍需语言或上下文模态的补充。

总体而言,头部姿态为实时生成提供了介于韵律与语义之间的关键中层约束。%
其时间上的前瞻性与空间上的指向性共同帮助模型在低延迟条件下保持自然、连贯且空间协调的动作表现,%
从而在因果生成框架内有效拓展了语音驱动手势的可表达范围,并为节奏主导型动作的实时生成提供了结构的支持。

基于上述任务范围与模态设计原则,下一节将进一步给出本文系统的总体架构,并说明各模块在实时生成流程中的功能定位。