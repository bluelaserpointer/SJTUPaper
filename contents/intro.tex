% !TEX root = ../main.tex

\chapter{引言}

\section{研究背景和意义}

近年来,随着元宇宙、虚拟社交与直播等领域的相关技术日趋成熟,用户已能够使用任意外观的虚拟人作为交互载体,在虚拟空间中与异地用户进行交流。虚拟人3D模型的姿态由其内部骨架关节的旋转参数(如欧拉角、四元数等)定义,最终通过蒙皮渲染技术完成可视化。得益于自动骨骼绑定技术,骨骼动画的生成可消除不同3D模型间的骨架拓扑差异,实现跨模型复用。

在虚拟人交互中,穿戴式动作捕捉设备是实时驱动手势的传统方案,贴合肢体的标记点可将肢体运动实时转换为相同的骨骼动画,提供直观、准确的操控。尽管其精度较高,但对于大多数用户而言,此类设备存在功能用途单一、硬件成本昂贵、便携性差等问题,限制了使用频率。因此,在当前的虚拟人交互应用中,仅少数专业用户会使用此类设备,而多数用户在设备限制下无法简单控制虚拟人肢体,导致了两种用户体验之间的不一致。

针对普通用户对低门槛虚拟人交互的需求,基于相机的动作捕捉技术\cite{mediapipefacemesh,AppleARKitTrackingGuide}成为主流替代方案。该技术无需额外硬件,仅通过手机或电脑的内置相机即可实时捕捉用户动作,用于转化为骨骼动画。然而,该方案仍存在三种局限:一是这种方法要求用户在空中面向相机做出手势,过程中难以同步操作键盘、鼠标,造成操作冲突;二是手势活动范围受相机视野限制,在自然使用距离通过手机或电脑的内置相机拍摄用户,将严重限制手势的捕捉范围;三是持续的手势动作会产生体力消耗,在直播等长时间使用场景中,用户疲劳问题将变得显著。”

因此,我们提出一种新的需求:一种无需用户实际做出手势,仅通过用户的实时语音、面部捕捉与头部姿态,实时生成与语义和情感相匹配的手势骨骼动画。该方法旨在降低使用门槛,并解决操作冲突与体力消耗问题。

然而,现有研究尚未提供成熟的解决方案满足上述需求。首先,当前多数手势生成方法依赖完整的语音或文本输入。由于用户的语音输入是逐字进行的,计算机需要等待用户的未来输入才能解析当前的语义,造成动画生成的延迟。其次,现有研究未对用户的头部姿态模态做深入研究。头部姿态对手势的节奏与朝向具有明确的关系,且可以通过常规相机实时捕捉,但利用该模态来增强生成手势的自然度的相关研究尚不充分。

为此,本文提出了一种新颖的实时手势生成模型。该模型以帧为单位,输入语音、面部表情与头部姿态数据,并逐帧输出对应的骨骼动画。本文首次在实时手势生成中,将头部姿态作为一种新的模态引入,利用其与自然手势节奏与朝向的高度相关性,结合语音和面部信息共同提升生成动作的自然度。我们采用级联多模态架构与自回归训练来融合这些模态,以学习其联合表征,从而在严格的实时约束下增强手势的表现力。

本文的主要贡献如下:

\begin{enumerate}
\item 提出了 FaceCapGes,一种帧级实时手势生成模型,使用户无需动作捕捉设备或实际做出手势,仅通过语音等常见输入即可驱动虚拟人的手势动画;
\item 将头部姿态作为新模态引入多模态级联架构,在实时生成约束下有效提升了手势的自然性与表现力;
\item 通过实验结果表明,该模型在手势自然性、语音-手势对齐度与实时响应方面展示了良好的性能。其框架适用于所有兼容 ARKit 的设备。
\end{enumerate}

\section{研究内容}

本文的研究目标是构建一种基于语音、面部捕捉与头部姿态的实时数字人手势生成模型,实现无需动作捕捉设备即可驱动虚拟角色自然表达的实时动画系统。该研究旨在解决现有手势生成方法对未来上下文的依赖及实时性不足的问题,从而为虚拟人交互提供更低门槛、更高沉浸度的解决方案。

为实现上述目标,本文的主要研究内容如下:

其一,设计一种帧级手势生成架构。
为实现帧级实时推理,本架构在基线模型 \cite {beatcamn} 的基础上做两种调整:一方面,保留其语音、面部表情等可实时获取的输入模态,移除非实时模态的输入分支与特征处理模块;另一方面,采用滑动窗口式自回归训练方式,确保模型仅依赖历史与当前帧信息进行推理(不依赖未来上下文),同时通过窗口内时序依赖建模保持动作的时间连续性。由此使架构可处理逐帧输入的实时流数据。

其二,提出一种引入头部姿态的新型模态融合策略。
在现有语音、面部表情生成手势的模型基础上,将头部姿态作为终端模态引入,设计头部姿态编码器并提取特征,以增强生成手势的朝向的自然性。

其三,搭建实验系统与用户测试环境。
本文基于主流渲染引擎构建了虚拟人驱动与手势动画的可视化系统。该系统作为评估平台,为后续的主观实验提供了统一环境。

综上所述,本文引入头部姿态的新模态,来辅助手势生成模型从历史与现在信息推理手势的能力,并且提供了一个实验场景验证模型的推理质量与实时性能。

\section{论文组织架构}

本文共分为六个章节,内容安排如下:

第一章为绪论,介绍本研究的背景与意义,阐述研究目标、主要内容及核心贡献,并说明论文的整体组织结构。

第二章为相关工作,回顾了国内外在语音驱动手势生成、多模态学习及实时面部捕捉技术等方面的研究现状,分析了现有方法的不足,并明确了本文的研究定位与创新点。

第三章为方法, 详细介绍了本文提出的 FaceCapGes 模型的整体架构与算法流程,包括头部姿态的多模态输入编码机制、姿态解码机制、滑动窗口自回归训练策略及对抗优化过程。

第四章为评估,详细介绍了客观评估结果与用户调研结果,对生成质量进行综合分析,并进一步通过消融实验与推理速度测试验证头部姿态输入与帧级自回归生成策略在实时交互场景下的有效性。

第五章为结论,总结本文的主要研究成果与贡献,讨论当前系统的局限性,并对未来在语义手势生成与未来趋势预测建模方向上的研究进行展望。
